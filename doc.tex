\batchmode
\documentclass[10pt,a4paper]{article}
\usepackage[utf8x]{inputenc}

%Warning: tipa declares many non-standard macros used by utf8x to
%interpret utf8 characters but extra packages might have to be added
%such as "textgreek" for Greek letters not already in tipa
%or "stmaryrd" for mathematical symbols.
%Utf8 codes missing a LaTeX interpretation can be defined by using
%\DeclareUnicodeCharacter{code}{interpretation}.
%Use coqdoc's option -p to add new packages or declarations.
\usepackage{tipa}

\usepackage[T1]{fontenc}
\usepackage{fullpage}
\usepackage{coqdoc}
\usepackage{amsmath,amssymb}
\usepackage{url}
\usepackage{graphicx}
\usepackage{bbm}
\usepackage{hyperref}
\usepackage{xspace}

\newcommand{\FSharp}{\textsf{F\nolinebreak[4]\kern-.05em\raisebox{.2ex}{\small\#}}\xspace}

\usepackage[parfill]{parskip}
\setlength{\coqdocbaseindent}{2em}

\let\oldCoqdocemptyline\coqdocemptyline
\let\oldCoqdoceol\coqdoceol
\renewcommand{\coqdocemptyline}{\par}
\renewcommand{\coqdoceol}{\oldCoqdoceol\let\coqdocemptyline\oldCoqdocemptyline}
\renewenvironment{coqdoccode}{\ignorespaces}{}

\renewcommand{\coqdoctac}{\coqdocvar}

\title{%
  Formal Specification of VM and I/O devices\\
  and description validation\\[2ex]
  \large \textbf{Objective 2.3} of the iVM project in the Eurostars Programme
}
\author{Ivar Rummelhoff}
\date{2019/11/01}

\begin{document}
\maketitle

\section{Introduction}

The focus of this document is a formal specification of iVM virtual machine corresponding to the natural language description given in O2.2. The specification has been formalized and mechanically checked using the \emph{Coq} proof assistant. The source code will be made publicly available at:

\begin{center}
  \url{https://github.com/preservationvm/formalizations}
\end{center}

The mathematical description in section~\ref{sec:formal} has been extracted from the Coq formalization. The description will be included on the Piql film together with the less formal description in O2.2 and a precise description of the physical I/O devices. We specify the \emph{interface} between these devices and the machine, but a formal specification of the physical devices themselves is beyond the scope of the current document.

The machine specification has been formulated in terms of \emph{monads}, a mathematical concept much used in computer science. As a result, the specification resembles an implementation of the machine in a functional programming language. This makes it easy to confirm that O2.2 describes the same machine. The description in section~\ref{sec:formal} assumes that the reader has a certain ``mathematical maturity''. Nevertheless, we include precise definitions of established concepts such as monads and monad transformers in order to reduce the risk of misunderstandings. We also explain the Coq syntax when it is not obvious from the context and point out when we diverge from ``standard Coq'' in order to simplify the presentation (e.g. writing $\mathbbm{1}$ instead of $\coqdocvar{unit}$ for the singleton type).

Our virtual machine is so simple that it is hard to write a C compiler targetting it directly. For this reason, we have defined an intermediate ``assembly language'' and an assembler which translates it into the machine language. Since this is a non-trivial transformation, we have written a number of automatic tests for it. Instead of checking the assembler output directly, these tests run the resulting machine code on a prototype implementation of the virtual machine written in \FSharp. Since this is a functional programming language, the prototype implementation is virtually identical to the specification below. Thus, the tests also confirm the ``sanity'' of the machine specification. These tests will be described in detail in O2.4.

\nonstopmode
\include{ivm}
\end{document}
